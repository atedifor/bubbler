\documentclass{bubbler}

\renewcommand{\bubblertitle}{%
\unbubskip
\bol{Alice} and \bol{Bob}\br
talk about\br
the brand new\br
\uli{\bol{BUBBLER}}
\rebubskip}
\renewcommand{\bubblercredit}{Created by}
\renewcommand{\bubblerauthor}{Eduard "Eddie" Forejt}
\renewcommand{\bubblersource}{\ita{Based on Fountain}}
\renewcommand{\bubblerrevision}{1.0}
\renewcommand{\bubblerdate}{May 2025}
\renewcommand{\bubblercopyright}{\cczero~atEdiFor}

\begin{document}
\begin{script}

\contop

\seg{ACT I - PREPARATIONS}

\scene[INT][]{ROSE CAFE}

Alice and Bob are sitting in a café. Bob is showing Alice the script he wrote on his computer.

\conv[][]{ALICE}{%
Woah, your script's so cool, Bob!}

\conv[][]{BOB}{%
Thanks, Alice.\br
Now that I've found a software that I can work with, writing's been a breeze.}

\conv[][]{ALICE}{%
What's it called?}

\conv[][]{BOB}{%
It's called Bubbler, it's a brand-new document class for LuaLaTeX based on Fountain.}

\conv[][]{ALICE}{%
I know Fountain, I've used it before. It has a more markdown-y feel to it, right?}

\conv[][]{BOB}{%
Yeah, TeX might seem more advanced compared to it, but with Bubbler it's not that hard.}

Alice leans closer to Bob's computer.

\conv[][]{ALICE}{%
Really? Can you show me?}

Bob nods.

\conv[][]{BOB}{%
Of course.}

\clearpage

\seg{ACT II - THE TITLE PAGE \& FORMATTING}

A waitress brings Bob and Alice their coffe.

\conv[][]{WAITRESS}{%
Here you go.}

Alice and Bob thank her and the waitress silently leaves.

\conv[][]{ALICE}{%
Let's see here. This title page for example, that must have been pretty hard to set up.}

\conv[][]{BOB}{%
Not at all. It's just the matter of figuring out what you want to place where. Bubbler comes with a lot of options by default, most of them inspired by Better Fountain. Then it's just a matter of using\br
\bol{\textbackslash renewcommand\{\textbackslash bubblertitle\}\{etc.\}}\br
You would replace\br
\bol{\textbackslash bubblertitle}\br
with the space you want to use and\br
\bol{etc.}\br
with the text you want to show in it.}

\conv[][]{ALICE}{%
But, if there are so many spaces, how can you remember them all?}

\conv[][]{BOB}{%
Well, I could check the code or the documentation but I don't really have to. With Bubbler, it's just a matter of using\br
\bol{\textbackslash titlehelp}\br
and all the spaces appear with their corresponding names.}

\conv[][]{ALICE}{%
That's neat.}

Bob takes a sip of his coffee.

\conv[][]{BOB}{%
It is.}

\conv[][]{ALICE}{%
But what if I don't want a title page because my script is still a prototype or doesn't have a name yet.}

\conv[][]{BOB}{%
Bubbler's ready for that too, you would just put\br
\bol{\textbackslash skiptitle}\br
in the preamble and the title page will not appear in the resulting script.}

\conv[][]{ALICE}{%
How about the formatting, the bold and italic fonts?}

\conv[][]{BOB}{%
For \bol{bold} text, you simply wrap it in\br
\bol{\textbackslash bol\{text\}}\br
For \ita{italics}, you use\br
\bol{\textbackslash ita\{text\}}\br
If you want \bolita{both}, you use\br
\bol{\textbackslash bolita\{text\}}\br
For \uli{underlined} text, it's\br
\bol{\textbackslash uli\{text\}}}

Alice smiles and nods.

\conv[][]{BOB}{%
Anything else?}

\conv[][]{ALICE}{%
How about forcibly-splitting paragraphs. If I recall correctly, it's really wrong to do it in TeX but I use it all the time in my scripts, for emphasis and such.}

\conv[][]{BOB}{%
Bubbler's got you covered. It still uses paragraphs, it just temporarily shrinks the space between them. You can either write a regular paragraph with a space after it and then use\br
\bol{\textbackslash joinup\{text\textbackslash br more text\}}\br
to, well, join it up to the paragraph. You don't use more paragraphs in this though, you replace them with\br
\bol{\textbackslash br}}

Alice listens while bringing up her own tablet.

\conv[][]{ALICE}{%
Alright, and looking at the code of Bubbler, you can also do it manually by using\br
\bol{\textbackslash unbubskip}\br
before the paragraph and\br
\bol{\textbackslash rebubskip}\br
after it ends.}

Bob nods.

\conv[][]{BOB}{%
Simple, right?}

\cut{CUT TO:}

\clearpage

\seg{ACT III - PAGE FORMAT \& CONTENTS}

\scene[EXT][]{BEFORE ROSE CAFE}

Bob and Alice are holding to-go cups and their respective digital divices.

\conv[][annoyed]{ALICE}{%
That's nice and all, but we still haven't talked about the script yet.}

\conv[][]{BOB}{%
Sorry, about that, but we might need to make one more detour before we can start working on our script.}

Alice sighs.

\conv[][]{ALICE}{%
Alright.}

\conv[][]{BOB}{%
I already told you that Bubbler works based on LuaLaTeX, which in turn is based upon LaTeX.}

\conv[][]{ALICE}{%
Yes? Which means?}

\conv[][]{BOB}{%
That we need to do some things before we start writing. First we need to start the document environment with\br
\bol{\textbackslash begin\{document\}}\br
and then inside it we start the script environment\br
\bol{\textbackslash begin\{script\}}\br
This prepares the title page and sets up some things for later.}

Alice murmurs aggreingly.

\conv[][]{BOB}{%
It's also a good idea to put\br
\bol{\textbackslash contop}\br
right before the start of the script's contents, as it's also used by things later on.}

\conv[][]{ALICE}{%
Is that all?}

\cut{FADE TO:}

\clearpage

\seg{ACT IV - SCRIPT WRITING}

\scene[EXT][]{PARK}

Alice and Bob walk into the nearby park.

\conv[][]{PASSERBY}{%
Hi, Alice.\br
Hello, Bob.}

They wave in answer and sit down on a bench.

\conv[][]{ALICE}{%
Come on, Bob, I want to get to writing.}

Bob chuckles.

\conv[][]{BOB}{%
Of course.\br
You already saw some of the formatting in my script, but let's take a closer look at how we get things to work.}

Bob opens his laptop again.

\conv[][]{BOB}{%
You already saw the conversation, you make these like this\br
\bol{\textbackslash conv[type][method]\{name\}\{text\}}\br
Name and text are the required arguments, name is for the character and text is where you put the speech.\br
You can leave out type and method if you want. Type is written after the character's name, it's usually something like O.S. for off-screen. Method is written under the name and serves as a way of telling how their voice or emotion should sound.}

\conv[][]{ALICE}{%
Sure, I know that. And I'm reading here that if you want to include an action in the conversation, you use\br
\bol{\textbackslash inconv\{action\}}}

\conv[][]{BOB}{%
You sure do.}

\dia{%
\diaconv[][]{ALICE}{%
Can you also do-}}{%
\diaconv[][]{BOB}{%
What about-}}

Alice and Bob laugh.

\conv[][]{BOB}{%
Sorry, you first.}

\conv[][]{ALICE}{%
Just wanted to ask if you can make two people talk at the same time but I've already figured it out.}

\conv[][]{BOB}{%
Yep, you use\br
\bol{\textbackslash dia\{\textbackslash diaconv\}\{\textbackslash diaconv\}}\br
where\br
\bol{\textbackslash diaconv}\br
works the same as regular conv, it's just prepared to fit inside the dialogue.}

\conv[][]{ALICE}{%
And you wanted to say?}

Bob looks up from his laptop.

\conv[][]{BOB}{%
What-\br
\inconv{pause, thinking}\br
Oh, right, scenes.}

\conv[][]{ALICE}{%
Yeah, you need those.}

\conv[][]{BOB}{%
It's quite simple as well.}

\conv[][quietly]{ALICE}{%
You say that about everything.}

\conv[][]{BOB}{%
Because it's true!\br
Scenes are made like this\br
\bol{\textbackslash scene[int/ext][daytime]\{name\}}}

\conv[][]{ALICE}{%
Let me guess, name is required while int/ext and daytime are optional.}

\conv[][]{BOB}{%
That's right, one important thing to note though is that they use special formatting.\br
Int/ext is suffixed with a period (.), which you don't need to include, and should write int/ext in a way that suits this.\br
On the other hand, daytime is prefixed with a dash (-).}

\conv[][]{ALICE}{%
And cuts are written as\br
\bol{\textbackslash cut\{CUT TO:\}}\br
and here you do need to put in the colon (:) as it isn't put in automatically.}

\conv[][]{BOB}{%
Last, but not least is music. For now, it's mostly just an alias for \ita{italics}, but it's still recommended to use the specific macros in case that changes in the future. The macros are\br
\bol{\textbackslash lyric\{singing goes here\}}\br
for a character singing inside their conversation block and\br
\bol{\textbackslash music\{track name\}}\br
for scene music.}

Alice slides a finger across her tablet's surface, reading through the Bubbler documentation available on its GitHub page.

\conv[][]{ALICE}{%
And finally you can do segments with\br
\bol{\textbackslash seg\{name\}}\br
and if you want to also put it on a new page you can use\br
\bol{\textbackslash cseg\{name\}}}

\conv[][]{BOB}{%
You can also center any other text by using\br
\bol{\textbackslash encen\{text\}}}

With the sky turning dark, Alice and Bob leave the park to walk home.

\cut{> CUT TO:}

\clearpage

\seg{ACT V - TOOLS}

\scene[EXT][NIGHT]{STREETS}

Alice is walking ahead of Bob, excited to write her script.

\conv[O.S.][]{BOB }{%
Wait for me!}

\conv[][excited]{ALICE}{%
Sorry, Bob, I'm just so looking forward to getting home and starting to write.}

Bob finally catches up to Alice.

\conv[][with heavy breath]{BOB}{%
Yeah, that's great, but there's still couple more things to tell you.}

Alice turns to Bob.

\conv[][]{ALICE}{%
Right, I meant to ask you. I'm actually writing a script that has to comply with the Oscar standards and I noticed that Bubble uses scene numbering and bold scene names, which are both against those standards. What-}

Bob cuts her off.

\conv[][]{BOB}{%
If you need to conform to Oscar standards, you just put\br
\bol{\textbackslash oscarstyle}\br
into your script's preamble. You would then also use\br
\bol{\textbackslash noscenenumber}\br
for scenes without numbers and\br
\bol{\textbackslash noboldscene}\br
for scenes with regular-weight names.}

Alice nods.

\conv[][]{ALICE}{%
Thanks. What else is there to tell?}

Bob smiles.

\conv[][]{BOB}{%
Bubbler also comes with couple tools that you can use to make your life easier.\br
First there is upbubble,\br
\bol{\textbackslash upbubble\{file.tex\}\{title\}}\br
which allows you to embed one script in Bubbler from\br
\bol{\textbackslash contop}\br
to\br
\bol{\textbackslash end\{script\}}\br
in another script.}

Alice is looking through the documentation, trying to find the part about tools.

\conv[][]{ALICE}{%
Here it is!\br
\inconv{pauses, reading*}\br
Seems like it can also automatically convert a Fountain file to Bubbler/TeX either with\br
\bol{\textbackslash rebubble[optional title]\{file.fountain\}}\br
or by using the attached Lua file like this\br
\bol{bubbler.lua source.fountain destination.tex}}

\conv[][]{EDDIE}{%
That's how I'm making this script!}

A random Eddie disappears as quickly as they appeared. Alice and Bob shrug, continuing to walk down the road.

\conv[][]{BOB}{%
You sure can. The next thing you can do is get a list of characters\br
\bol{\textbackslash dumpmem[optional title]}\br
which also resets the context or you can do that manually with\br
\bol{\textbackslash clearmem}\br
See? Like this.}

\dumpmem[.]

\clearpage

\conv[][]{ALICE}{%
Cool.}

\conv[][]{BOB}{%
But it has to be after the section from where you want to get the characters. Only after, not before.}

Alice and Bob approach their apartment's door.

\conv[][]{BOB}{%
One last thing. Remember how I said that beginning the script environment will be important later on?}

Alice nods.

\conv[][]{BOB}{%
Well, this is where it comes in. The beginning actually has three optional arguments.\br
\bol{\textbackslash begin[character limit][scene limit][scene categories]\{script\}}\br
This changes if and how detailed a statistics print-out will be.}

\conv[][]{ALICE}{%
A statistics print-out?}

\conv[][]{BOB}{%
Yes, at the end of the TeX output there can be a table of statistics, how much do scenes and characters appear in the script.\br
It is disabled by default though but you can enable it and customize it using the code above.\br
Character limit means how many of the top characters will be printed out.\br
Same with scene limit, but for scenes.\br
You can use -1 to print out all of them, respectively.\br
Scene categories can be either set to 0 (false) or 1 (true), which disables/enables (respectively). And -1 to disable all statistics.}

\cut{FADE OUT.}

\begin{comment}

Hi, you found me!
I'm in your comment section, muhahah, and can do some evil stuff like
\diaconv{\conv{}}{}

\end{comment}
\end{script}
\end{document}
